\documentclass[a4paper,norsk,12pt]{article}
\usepackage[utf8]{inputenc}

% Oppsett for norsk
\usepackage[norsk]{babel}
\usepackage{times}
\usepackage[T1]{fontenc}
\usepackage{parskip}
\DeclareUnicodeCharacter{00A0}{ }
\newcommand{\strek}{\textthreequartersemdash}

% Andre pakker
\usepackage{oving}
\usepackage{amsmath}
\usepackage{amssymb}
\usepackage{varioref}
\usepackage{subcaption}
\usepackage{units}


\title{MAT100 --- Matematiske metoder 1}
\subtitle{Obligatorisk innlevering 3}
\author{Christian Stigen}
\date{UiS, 30.~oktober, 2015}

\begin{document}
\maketitle

\oppgave{1a}
Da $\ln{x}' = \frac{1}{x}$ ikke inneholder $\ln{x}$ prøver vi oss med
produktregelen,
\begin{align*}
  \left( x \ln{x} \right)' = \ln{x} + x\frac{1}{x} = \ln{x} + 1
\end{align*}
Dette er én mer enn vi er ute etter, men vi vet at $\int 1 \,dx = x$, dermed
prøver vi
\begin{align*}
  \left( x \ln{x} - x\right)' = \ln{x} + 1 - 1 = \ln{x}
\end{align*}
Altså får vi
\begin{align*}
  \int \ln{x} \,dx = \svar{x\ln{x} - x + C}
\end{align*}


\oppgave{1b}
Vi prøver
\begin{align*}
 \left( x e^{2x} \right)' = e^{2x} + 2x e^{2x}
\end{align*}
For å eliminere leddet $e^{2x}$ er det nok å trekke fra
$\frac{1}{2}e^{2x}$.
\begin{align*}
  \left( x e^{2x} - \frac{1}{2}e^{2x} \right)' = 2x e^{2x}
\end{align*}
Dette er det dobbelte av det vi er ute etter, dermed kan vi multiplisere hele
uttrykket med en halv, og får
\begin{align*}
  \int x e^{2x} \,dx = \frac{1}{2}\left( x e^{2x} - \frac{1}{2} e^{2x} \right)
  + C
  = \svar{\frac{e^{2x}}{4}(2x - 1) + C}
\end{align*}


\oppgave{1c}
Vi vet at $\frac{d}{dx}\ln{x} = \frac{1}{x}$. Dermed kan vi bruke
kjerneregelen:
\begin{align*}
  g(x) &= \ln{x} \\
  f(x) &= x^2+1 \\
  h(x) &= g(f(x)) = \ln{(x^2+1)} \\
  h'(x) &= g'(f(x)) \cdot f'(x) = \frac{1}{x^2+1} \cdot 2x
\end{align*}
Dette gir direkte
\begin{align*}
  \int \frac{2x}{x^2+1} \,dx = \svar{\ln{(x^2+1)} + C}
\end{align*}

\oppgave{1d}
Vi vet at $\frac{d}{dx}\sqrt{x} = \frac{1}{2\sqrt{x}}$, dermed kan vi prøve å
bruke kjerneregelen igjen:
\begin{align*}
  g(x) &= \sqrt{x} \\
  f(x) &= e^x+1 \\
  h(x) &= g(f(x)) = \sqrt{e^x+1} \\
  h'(x) &= g'(f(x)) \cdot f'(x) = \frac{1}{2\sqrt{e^x+1}} \cdot e^x
\end{align*}
Dette er halvparten av det vi er ute etter, dermed kan vi skrive
\begin{align*}
  \int \frac{e^x}{\sqrt{e^x+1}} \,dx = \svar{2\sqrt{e^x+1} + C}
\end{align*}

\oppgave{1e}
Vi vet at $(x^4)' = 4x^3$. Vi bruker kjerneregelen med $\frac{1}{4}$ som
kansellering av faktoren fire:
\begin{align*}
  g(x) &= \frac{1}{4} x^4 \\
  f(x) &= \sin{x} \\
  h(x) &= g(f(x)) = \frac{1}{4}(\sin{x})^4 = \frac{1}{4} \sin^4{x} \\
  h'(x) &= g'(f(x)) \cdot f'(x) = \sin^3{x} \cdot \cos{x}
\end{align*}
Altså er
\begin{align*}
  \int \sin^3{x} \cos{x} \,dx = \svar{ \frac{1}{4} \sin^4{x} + C }
\end{align*}

\oppgave{1f}
\textit{Ikke utført}

\oppgave{1g}
Vi prøver oss litt frem med derivasjon.
\begin{align*}
  (e^x \cos{x})' &= e^x\cos{x} - e^x\sin{x} \\
  (e^x \sin{x})' &= e^x\sin{x} + e^x\cos{x}
\end{align*}
Ved å summere uttrykkene over så kanselleres $e^x\sin{x}$. Det gir oss
imidlertid $2e^x\cos{x}$, derfor prøver vi å derivere summen multiplisert med
en halv:
\begin{align*}
  \left( \frac{1}{2}e^x \cos{x} +
         \frac{1}{2}e^x \sin{x} \right)' =
    \frac{1}{2} \left( 
        e^x\cos{x}-e^x\sin{x} + e^x\sin{x} + e^x\cos{x}
      \right)' = e^x\cos{x}
\end{align*}
Dermed får vi
\begin{align*}
  \int{e^x\cos{x}} \,dx = 
  \svar{ \frac{1}{2}e^x( \sin{x} + \cos{x} ) + C }
\end{align*}


\oppgave{2a}
\textit{Ikke utført}

\oppgave{2b}
\textit{Ikke utført}

\oppgave{2c}
\textit{Ikke utført}

\oppgave{2d}
\textit{Ikke utført}

\oppgave{3a}
\begin{align*}
  f'(x) = 1 + 2\cos{x} & = 0\\
  \cos{x} & = -\frac{1}{2}
\end{align*}

\oppgave{3b}
\textit{Ikke utført}

\oppgave{4}
\textit{Ikke utført}

\oppgave{5}
\textit{Ikke utført}

\end{document}
