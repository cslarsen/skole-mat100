\documentclass[a4paper,norsk,12pt]{article}
\usepackage[utf8]{inputenc}

% Oppsett for norsk
\usepackage[norsk]{babel}
\usepackage{times}
\usepackage[T1]{fontenc}
\usepackage{parskip}
\DeclareUnicodeCharacter{00A0}{ }
\newcommand{\strek}{\textthreequartersemdash}

% Andre pakker
\usepackage{oving}
\usepackage{amsmath}
\usepackage{amssymb}
\usepackage{varioref}
\usepackage{subcaption}
\usepackage{units}


\title{MAT100 --- Matematiske metoder 1}
\subtitle{Obligatorisk innlevering 3}
\author{Christian Stigen}
\date{UiS, 30.~oktober, 2015}

\begin{document}
\maketitle

\oppgave{1a}
Da $\ln{x}' = \frac{1}{x}$ ikke inneholder $\ln{x}$ prøver vi oss med
produktregelen,
\begin{align*}
  \left( x \ln{x} \right)' = \ln{x} + x\frac{1}{x} = \ln{x} + 1
\end{align*}
Dette er én mer enn vi er ute etter, men vi vet at $\int 1 \,dx = x$, dermed
prøver vi
\begin{align*}
  \left( x \ln{x} - x\right)' = \ln{x} + 1 - 1 = \ln{x}
\end{align*}
Altså får vi
\begin{align*}
  \int \ln{x} \,dx = \svar{x\ln{x} - x + C}
\end{align*}


\oppgave{1b}
Vi prøver
\begin{align*}
 \left( x e^{2x} \right)' = e^{2x} + 2x e^{2x}
\end{align*}
For å eliminere leddet $e^{2x}$ er det nok å trekke fra
$\frac{1}{2}e^{2x}$.
\begin{align*}
  \left( x e^{2x} - \frac{1}{2}e^{2x} \right)' = 2x e^{2x}
\end{align*}
Dette er det dobbelte av det vi er ute etter, dermed kan vi multiplisere hele
uttrykket med en halv, og får
\begin{align*}
  \int x e^{2x} \,dx = \frac{1}{2}\left( x e^{2x} - \frac{1}{2} e^{2x} \right)
  + C
  = \svar{\frac{e^{2x}}{4}(2x - 1) + C}
\end{align*}


\oppgave{1c}
\textit{Ikke utført}

\oppgave{1d}
\textit{Ikke utført}

\oppgave{1e}
\textit{Ikke utført}

\oppgave{1f}
\textit{Ikke utført}

\oppgave{1g}
\textit{Ikke utført}

\oppgave{2a}
\textit{Ikke utført}

\oppgave{2b}
\textit{Ikke utført}

\oppgave{2c}
\textit{Ikke utført}

\oppgave{2d}
\textit{Ikke utført}

\oppgave{3a}
\textit{Ikke utført}

\oppgave{3b}
\textit{Ikke utført}

\oppgave{4}
\textit{Ikke utført}

\oppgave{5}
\textit{Ikke utført}

\end{document}
