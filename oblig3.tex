\documentclass[a4paper,norsk,12pt]{article}
\usepackage[utf8]{inputenc}

% Oppsett for norsk
\usepackage[norsk]{babel}
\usepackage{times}
\usepackage[T1]{fontenc}
\usepackage{parskip}
\DeclareUnicodeCharacter{00A0}{ }
\newcommand{\strek}{\textthreequartersemdash}

% Andre pakker
\usepackage{oving}
\usepackage{amsmath}
\usepackage{amssymb}
\usepackage{varioref}
\usepackage{subcaption}
\usepackage{units}


\title{MAT100 --- Matematiske metoder 1}
\subtitle{Obligatorisk innlevering 3}
\author{Christian Stigen}
\date{UiS, 30.~oktober, 2015}

\begin{document}
\maketitle

\oppgave{1a}
Da $\ln{x}' = \frac{1}{x}$ ikke inneholder $\ln{x}$ prøver vi oss med
produktregelen,
\begin{align*}
  \left( x \ln{x} \right)' = \ln{x} + x\frac{1}{x} = \ln{x} + 1
\end{align*}
Dette er én mer enn vi er ute etter, men vi vet at $\int 1 \,dx = x$, dermed
prøver vi
\begin{align*}
  \left( x \ln{x} - x\right)' = \ln{x} + 1 - 1 = \ln{x}
\end{align*}
Altså får vi
\begin{align*}
  \int \ln{x} \,dx = \svar{x\ln{x} - x + C}
\end{align*}


\oppgave{1b}
Vi prøver
\begin{align*}
 \left( x e^{2x} \right)' = e^{2x} + 2x e^{2x}
\end{align*}
For å eliminere leddet $e^{2x}$ er det nok å trekke fra
$\frac{1}{2}e^{2x}$.
\begin{align*}
  \left( x e^{2x} - \frac{1}{2}e^{2x} \right)' = 2x e^{2x}
\end{align*}
Dette er det dobbelte av det vi er ute etter, dermed kan vi multiplisere hele
uttrykket med en halv, og får
\begin{align*}
  \int x e^{2x} \,dx = \frac{1}{2}\left( x e^{2x} - \frac{1}{2} e^{2x} \right)
  + C
  = \svar{\frac{e^{2x}}{4}(2x - 1) + C}
\end{align*}


\oppgave{1c}
Vi vet at $\frac{d}{dx}\ln{x} = \frac{1}{x}$. Dermed kan vi bruke
kjerneregelen:
\begin{align*}
  g(x) &= \ln{x} \\
  f(x) &= x^2+1 \\
  h(x) &= g(f(x)) = \ln{(x^2+1)} \\
  h'(x) &= g'(f(x)) \cdot f'(x) = \frac{1}{x^2+1} \cdot 2x
\end{align*}
Dette gir direkte
\begin{align*}
  \int \frac{2x}{x^2+1} \,dx = \svar{\ln{(x^2+1)} + C}
\end{align*}

\oppgave{1d}
Vi vet at $\frac{d}{dx}\sqrt{x} = \frac{1}{2\sqrt{x}}$, dermed kan vi prøve å
bruke kjerneregelen igjen:
\begin{align*}
  g(x) &= \sqrt{x} \\
  f(x) &= e^x+1 \\
  h(x) &= g(f(x)) = \sqrt{e^x+1} \\
  h'(x) &= g'(f(x)) \cdot f'(x) = \frac{1}{2\sqrt{e^x+1}} \cdot e^x
\end{align*}
Dette er halvparten av det vi er ute etter, dermed kan vi skrive
\begin{align*}
  \int \frac{e^x}{\sqrt{e^x+1}} \,dx = \svar{2\sqrt{e^x+1} + C}
\end{align*}

\oppgave{1e}
Vi vet at $(x^4)' = 4x^3$. Vi bruker kjerneregelen med $\frac{1}{4}$ som
kansellering av faktoren fire:
\begin{align*}
  g(x) &= \frac{1}{4} x^4 \\
  f(x) &= \sin{x} \\
  h(x) &= g(f(x)) = \frac{1}{4}(\sin{x})^4 = \frac{1}{4} \sin^4{x} \\
  h'(x) &= g'(f(x)) \cdot f'(x) = \sin^3{x} \cdot \cos{x}
\end{align*}
Altså er
\begin{align*}
  \int \sin^3{x} \cos{x} \,dx = \svar{ \frac{1}{4} \sin^4{x} + C }
\end{align*}

\oppgave{1f}
Vi skriver om
\begin{align}
  \label{1f.1}
  \frac{x+8}{x^3+4x} = \frac{x+8}{x(x^2+4)} =
    \frac{1}{x^2+4} + 8\frac{1}{x}\frac{1}{x^2+4}
\end{align}
Fra før vet vi at $\frac{d}{dx}\tan^{-1}{x} = \frac{1}{x^2+1}$. Dersom vi
bruker kjerneregelen med $f(x) = \frac{1}{2}x$, $g(x) = \tan^{-1}{x}$ og $h(x) =
g(f(x)) = \tan^{-1}{\frac{x}{2}}$ så får vi
\begin{align*}
  \left( \tan^{-1}{\frac{x}{2}} \right)' =
    \frac{1}{\frac{1}{4}x^2+1}\cdot\frac{1}{2} = \frac{2}{x^2+4}
\end{align*}
Dermed ser vi at
\begin{align}
  \label{1f.x3}
  \int\frac{1}{x^2+4}\,dx = \frac{1}{2}\tan^{-1}{\frac{x}{2}} + C
\end{align}
%
Videre vet vi at
\begin{align}
  \label{1f.x1}
  (2\ln{x})' &= \frac{2}{x} \\
  \label{1f.x2}
  (\ln{x^2+1})' &= \frac{2x}{x^2+1} \text{~fra oppgave 1c}
\end{align}
Ved å ta $(\ln{x^2+4})' = \frac{2x}{x^2+4}$
så kan vi kombinere disse to til å få siste ledd på høyresiden i likning
\vref{1f.1} over.
\begin{align}
  \label{1f.2}
  \frac{2}{x} - \frac{2x}{x^2+4} = \frac{2(x^4+4)}{x(x^2+4)}
    - \frac{2x^2}{x(x^2+4)} 
    = \frac{(2x^2+8) - 2x^2}{x(x^2+4)} = \frac{8}{x(x^2+4)}
\end{align}
Venstresiden i likning \vref{1f.2} over er triviell å integrere --- vi bruker
likningene \vref{1f.x3}, \vref{1f.x1} og \vref{1f.x2} (med kjernen $x^2+4$)
over, og får til slutt
\begin{align*}
  \int \frac{x+8}{x^3+4x} \,dx = \svar{\frac{1}{2}\tan^{-1}{\frac{x}{2}} +
  \ln{(x^2+4)} - 2\ln{x} + C}
\end{align*}

\oppgave{1g}
Vi prøver oss litt frem med derivasjon.
\begin{align*}
  (e^x \cos{x})' &= e^x\cos{x} - e^x\sin{x} \\
  (e^x \sin{x})' &= e^x\sin{x} + e^x\cos{x}
\end{align*}
Ved å summere uttrykkene over så kanselleres $e^x\sin{x}$. Det gir oss
imidlertid $2e^x\cos{x}$, derfor prøver vi å derivere summen multiplisert med
en halv:
\begin{align*}
  \left( \frac{1}{2}e^x \cos{x} +
         \frac{1}{2}e^x \sin{x} \right)' =
    \frac{1}{2} \left( 
        e^x\cos{x}-e^x\sin{x} + e^x\sin{x} + e^x\cos{x}
      \right)' = e^x\cos{x}
\end{align*}
Dermed får vi
\begin{align*}
  \int{e^x\cos{x}} \,dx = 
  \svar{ \frac{1}{2}e^x( \sin{x} + \cos{x} ) + C }
\end{align*}


\oppgave{2a}
\begin{align*}
  f(x) = e^\frac{x}{2} - 2 &= 0 \\
  \frac{x}{2} &= \ln{2} \\
  x &= \svar{ 2\ln{2} }
\end{align*}

\oppgave{2b}
Funksjonen øker eksponensielt mot høyre, men har en asymptote mot venstre. Vi
tar
\begin{align*}
  \lim_{x\to -\infty}{\left( e^\frac{x}{2}-2 \right)} = 0 - 2 = \svar{-2}
\end{align*}
Dette fordi $a^{-x} = \frac{1}{a^x}$ går mot null når $x\to\infty$.

\oppgave{2c}
\textit{Ikke fullstendig utført.}
Vi vet at $e^x$ vokser eksponensielt mot høyre. Mot venstre går den mot $-2$.
Den krysser y-aksen i punktet $e^0-2 = -1$ og krysser x-aksen i $2\ln{2}$. Ved
å regne ut $f(x)$ for noen punkter til høyre for nullpunktet kan vi skissere
grafen enkelt.

\oppgave{2d}
\begin{align*}
  \int_{2\ln{2}}^{4}f(x)\,dx
\end{align*}
Det ubestemte integralet er
\begin{align*}
  \int f(x) \,dx = 2e^\frac{x}{2} - 2x + C
\end{align*}
Vi ser bort fra konstanten $C$ som kanselleres.
\begin{align*}
  g(4) - g(2\ln{2}) & = (2e^2-8) - (4 - 4\ln{2}) \\
                    &= \svar{2e^2 - 12 + \ln{2^4}}
\end{align*}

\oppgave{3a}
\begin{align*}
  f(x) &= x+2\sin{x} \text{ for } x \in \left[-\pi, \frac{5\pi}{3}\right] \\
  f'(x) &= 1 + 2\cos{x} \\
  f''(x) &= -2\sin{x}
\end{align*}
Vi ser først på fortegnet til $f''(x)$ for å bestemme om lokale
ekstremalpunkter er maskimale eller minimale.
\begin{table}[h]
  \centering
\begin{tabular}{ccccccccc}
  & $-\pi$
  &
  & $0$
  &
  & $\pi$
  &
  & $2\pi$
  &
\\ \hline
    \textbf{\large{+}}
& & \textbf{\large{-}}
& & \textbf{\large{+}}
& & \textbf{\large{-}}
& & \textbf{\large{+}}
\end{tabular}
\end{table}

Punktene er gitt ved
\begin{align*}
  f'(x) = 0 \\
  \cos{x} = -\frac{1}{2}
\end{align*}
Da $\cos{x} = -\frac{1}{2}$ for $x=\frac{\pi}{3}$ må også
$\cos{x}=-\frac{1}{2}$ for alle $x = \frac{\pi}{3} + k\pi, k \in \mathbb{Z}$.
Dette fordi cosinus-kurven inntar alle verdier mellom $-1$ og $1$ \textit{to
ganger} i perioden $2\pi$. Dermed får vi følgende \textit{lokale} ekstremalpunkter
(altså, ikke medregnet endepunktene) for $x \in D_f$:
\begin{table}[h]
  \centering
  \begin{tabular}{ll}
    $x_0 = -\frac{2\pi}{3}$ & lokalt \textit{minimum} \\
    $x_1 =  \frac{2\pi}{3}$ & lokalt \textit{maksimum} \\
    $x_2 =  \frac{4\pi}{3}$ & lokalt \textit{minimum}
  \end{tabular}
\end{table}

\oppgave{3b}
Hvis vi regner ut $f(x)$ for alle ekstremalpunkter og endepunktene, så ser vi
at $x_0$ er absolutt minimum og $x=\frac{5\pi}{3}$ absolutt maksimum.
\begin{align*}
  f(-\pi) &= -\pi + 2\sin{\pi} = -\pi \\
  f(-\frac{2\pi}{3}) &= -\frac{2\pi}{3} - \sqrt{3} \text{~ ~  absolutt minimum} \\
  f(\frac{2\pi}{3}) &= \frac{2\pi}{3} + \sqrt{3} \\
  f(\frac{4\pi}{3}) &= \frac{4\pi}{3} - \sqrt{3} \\
  f(\frac{5\pi}{3}) &= \frac{5\pi}{3} - \sqrt{3} \text{~ ~ absolutt maksimum}
\end{align*}

\oppgave{4}
Volumet for en kule er $\frac{4}{3}\pi r^3$. Med $[v(t)] = \unit{m^3}$ har vi
at $v'(t) = \frac{1}{10}$ og dermed $v(t) = \frac{t}{10} + C$.
For å forenkle setter vi $v(0)=0$ slik at $C=0$. Da har vi
\begin{align*}
  v(t) &= \frac{t}{10} = \frac{4}{3}\pi r^3 \\
  r(t) &= \operatorname{Re}\left( \frac{3}{4\pi}~v(t) \right)^\frac{1}{3}
        = \operatorname{Re}\left( \frac{3}{4\pi}~\frac{t}{10} \right)^\frac{1}{3} \\
       &= \frac{1}{2} \operatorname{Re}\left( \frac{3}{5\pi}~t \right)^\frac{1}{3} \\
        r'(t) &= \frac{1}{6} \operatorname{Re}\left(
        (\frac{3}{5\pi})^\frac{1}{3} ~t^{-\frac{2}{3}}
          \right)
\end{align*}

\oppgave{5}
\textit{Ikke utført}

\end{document}
