\documentclass[a4paper,norsk,12pt]{article}
\usepackage[utf8]{inputenc}

% Oppsett for norsk
\usepackage[norsk]{babel}
\usepackage{times}
\usepackage[T1]{fontenc}
\usepackage{parskip}
\DeclareUnicodeCharacter{00A0}{ }
\newcommand{\strek}{\textthreequartersemdash}

% Andre pakker
\usepackage{oving}
\usepackage{amsmath}
\usepackage{amssymb}
\usepackage{varioref}
\usepackage{subcaption}
\usepackage{units}


\title{MAT100 --- Matematiske metoder 1}
\subtitle{Obligatorisk innlevering 1}
\author{Christian Stigen}
\date{UiS, 28.~september, 2015}

\begin{document}
\maketitle

\oppgave{1a}

\paragraph{(1)}
Med $i^2 = -1$ får vi
\begin{math}
  (1+i)(1+6i) = 1+6i+i-6 = \svar{-5+7i}
\end{math}

\paragraph{(2)}
Utvider brøken med $x_2 - iy_2 = 6 + 2i$ slik at divisor blir $x_2^2 +
y_2^2 = 40$,
\begin{equation*}
  \frac{(2+3i)(6+2i)}{40} = \frac{12+4i+18i-6}{40}
  = \frac{6+22i}{40} = \svar{\frac{3}{20} + \frac{11}{20}i}
\end{equation*}

\oppgave{1b}

Vi bruker $r = |z| = \sqrt{x^2+y^2}$ og $\theta = \text{arg } z =
\arctan\frac{y}{x}$ for å finne $z = re^{i\theta}$.
\begin{equation*}
\begin{split}
  r = \sqrt{3^2+3^2} \approx 4.24 \text{ og } \theta = \arctan{1} =
  \frac{\pi}{4} = 45^\circ \text{ gir } \svar{3+3i = 4.24 e^{i\frac{\pi}{4}}}
  \\
  r=\sqrt{3+1}=2 \text{ og } \theta = \arctan{-1/\sqrt{3}} = -30^\circ \approx -0.52
  \text{ gir }
  \svar{\sqrt{3}-i = 2e^{-0.52i}}
\end{split}
\end{equation*}
Se figur \vref{plot.1b} for plott av punktene i det komplekse planet.
%
\begin{figure}
\centering
\includegraphics[width=\textwidth]{plot.eps}
\caption{Plott av punkter i komplekst plan (oppgave 1b)}
\label{plot.1b}
\end{figure}

\oppgave{1c}
$z^c = e^{c\ln{z}}$ gir $\sqrt[3]{8i} = e^{\frac{1}{3}\ln{8i}}$.
\begin{equation*}
\begin{split}
\end{split}
\end{equation*}

\oppgave{2}
\oppgave{3}

\oppgave{4}
At det finnes minst én $x$ slik at $\cos{x}=x$ er det samme som å si at det
finnes et \textit{fikspunkt} for $\cos{x}$.

Vi vet at $-1 \leq \cos{x} \leq 1$ og at $\cos{0} = 1$ og $\cos{\frac{\pi}{2}}
= 0$. Samtidig vil $f(x)=x$ i $0$ være $f(0) = 0 < \cos{0}$ og
$f(\frac{\pi}{2}) = \frac{pi}{2} > \cos{\frac{\pi}{2}}$. Med andre ord
\textit{må} disse to grafene krysse hverandre, fordi de begge er kontinuerlige.
Det betyr at mellom $x=0$ og $x=\frac{\pi}{2}$ så må $\cos{x} = x$.

\oppgave{5}

\oppgave{6a}
Dersom vi utvider uttrykket med $\frac{1}{x}$ kan vi finne grenseverdien.
\begin{equation*}
\begin{split}
  \lim_{x \to -2}{\frac{x^2+4x+4}{x-2}} = \lim_{x \to -2}\frac{x+4+\frac{4}{x}}{1-\frac{2}{x}}
\end{split}
\end{equation*}
Vi setter inn $x=-2$ og får
\begin{equation*}
\begin{split}
  \lim_{x \to -2}\frac{-2+4+\frac{4}{-2}}{1-\frac{2}{-2}} =
  \frac{0}{2} = \svar{0}
\end{split}
\end{equation*}

\end{document}
