\documentclass[a4paper,norsk,12pt]{article}
\usepackage[utf8]{inputenc}

% Oppsett for norsk
\usepackage[norsk]{babel}
\usepackage{times}
\usepackage[T1]{fontenc}
\usepackage{parskip}
\DeclareUnicodeCharacter{00A0}{ }
\newcommand{\strek}{\textthreequartersemdash}

% AMS Math
\usepackage{amsmath}

\usepackage{oving}
\usepackage{varioref}


\title{MAT100 --- Matematiske metoder 1}
\subtitle{Obligatorisk innlevering 1}
\author{Christian Stigen}
\date{UiS, 28.~september, 2015}

\begin{document}
\maketitle

\oppgave{1a}

\paragraph{(1)}
Med $i^2 = -1$ får vi
\begin{math}
  (1+i)(1+6i) = 1+6i+i-6 = \svar{-5+7i}
\end{math}

\paragraph{(2)}
Utvider brøken med $x_2 - iy_2 = 6 + 2i$ slik at divisor blir $x_2^2 +
y_2^2 = 40$,
\begin{equation*}
  \frac{(2+3i)(6+2i)}{40} = \frac{12+4i+18i-6}{40}
  = \frac{6+22i}{40} = \svar{\frac{3}{20} + \frac{11}{20}i}
\end{equation*}

\oppgave{1b}

Vi bruker $r = |z| = \sqrt{x^2+y^2}$ og $\theta = \text{arg } z =
\arctan\frac{y}{x}$ for å finne $z = re^{i\theta}$.
\begin{equation*}
\begin{split}
  r = \sqrt{3^2+3^2} \approx 4.24 \text{ og } \theta = \arctan{1} =
  \frac{\pi}{4} = 45^\circ \text{ gir } \svar{3+3i = 4.24 e^{i\frac{\pi}{4}}}
  \\
  r=\sqrt{3+1}=2 \text{ og } \theta = \arctan{-1/\sqrt{3}} = -30^\circ \approx -0.52
  \text{ gir }
  \svar{\sqrt{3}-i = 2e^{-0.52i}}
\end{split}
\end{equation*}
Se figur \vref{plot.1b} for plott av punktene i det komplekse planet.

\begin{figure}
\centering
\includegraphics[width=\textwidth]{plot.eps}
\caption{Plott av punkter i komplekst plan (oppgave 1b)}
\label{plot.1b}
\end{figure}


\oppgave{2}
\oppgave{3}
\oppgave{4}
\oppgave{5}
\oppgave{6}

\end{document}
