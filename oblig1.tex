\documentclass[a4paper,norsk,12pt]{article}
\usepackage[utf8]{inputenc}

% Oppsett for norsk
\usepackage[norsk]{babel}
\usepackage{times}
\usepackage[T1]{fontenc}
\usepackage{parskip}
\DeclareUnicodeCharacter{00A0}{ }
\newcommand{\strek}{\textthreequartersemdash}

% Andre pakker
\usepackage{oving}
\usepackage{amsmath}
\usepackage{amssymb}
\usepackage{varioref}
\usepackage{subcaption}
\usepackage{units}


\title{MAT100 --- Matematiske metoder 1}
\subtitle{Obligatorisk innlevering 1}
\author{Christian Stigen}
\date{UiS, 28.~september, 2015}

\begin{document}
\maketitle

\oppgave{1a}

\paragraph{(1)}
Med $i^2 = -1$ får vi
\begin{math}
  (1+i)(1+6i) = 1+6i+i-6 = \svar{-5+7i}
\end{math}

\paragraph{(2)}
Utvider brøken med $x_2 - iy_2 = 6 + 2i$ slik at nevner blir $x_2^2 +
y_2^2 = 40$,
\begin{equation*}
  \frac{(2+3i)(6+2i)}{40} = \frac{12+4i+18i-6}{40}
  = \frac{6+22i}{40} = \svar{\frac{3}{20} + \frac{11}{20}i}
\end{equation*}

\oppgave{1b}

Vi bruker $r = |z| = \sqrt{x^2+y^2}$ og $\theta = \text{arg } z =
\tan^{-1}\frac{y}{x}$ for å finne $z = re^{i\theta}$.
\begin{equation*}
\begin{split}
  r = \sqrt{3^2+3^2} \approx 4.24 \text{ og } \theta = \tan^{-1}{1} =
  \frac{\pi}{4} = 45^\circ \text{ gir } \svar{3+3i = 4.24 e^{i\frac{\pi}{4}}}
  \\
  r=\sqrt{3+1}=2 \text{ og } \theta = \tan^{-1}{-1/\sqrt{3}} = -30^\circ \approx -0.52
  \text{ gir }
  \svar{\sqrt{3}-i = 2e^{-0.52i}}
\end{split}
\end{equation*}
%
Se figur \vref{plot.1b} for plott av punktene i det komplekse planet.

\begin{figure}
\centering
\includegraphics[width=0.8\textwidth]{plot.eps}
\caption{Plott av punkter i komplekst plan (oppgave 1b)}
\label{plot.1b}
\end{figure}

\oppgave{1c}
For $z=8i$ har vi $r = |z| = 8$, $\sqrt[3]{r} = (2^3)^\frac{1}{3} = 2$ og
$\text{Arg }z = \theta = \frac{\pi}{2}$ ($\tan^{-1}{n} \to \frac{\pi}{2} \text{ når } n \to
\infty$). Røttene under er plottet i figur \vref{plot.1c}.
%
\begin{equation*}
\begin{split}
  \sqrt[3]{8i} = 2\left( \cos{\frac{\theta+2k\pi}{3}} +
  i\sin{\frac{\theta+2k\pi}{3}} \right) \text{ for } k = 0, 1, 2
\end{split}
\end{equation*}
%
\begin{equation*}
\begin{split}
  k = 0 \text{ gir } w_0 & = 1.732 + 0.999i \\
  k = 1 \text{ gir } w_1 & = -1.732 + 0.999i \\
  k = 2 \text{ gir } w_2 & = -3.674 - 2.0i \\
\end{split}
\end{equation*}

\begin{figure}
\centering
\includegraphics[width=0.8\textwidth]{plot1c.eps}
\caption{Røtter for $\sqrt[3]{8i}$ (oppgave 1c)}
\label{plot.1c}
\end{figure}

\oppgave{2}
\oppgave{3}

\oppgave{4}
At det finnes minst én $x$ slik at $\cos{x}=x$ er det samme som å si at det
finnes et \textit{fikspunkt} for $\cos{x}$.

Vi vet at $-1 \leq \cos{x} \leq 1$ og at $\cos{0} = 1$ og $\cos{\frac{\pi}{2}}
= 0$. Samtidig har vi for $f(x)=x$ at $f(0) = 0 < \cos{0}$ og $f(\frac{\pi}{2})
= \frac{\pi}{2} > \cos{\frac{\pi}{2}}$. Med andre ord \textit{må} disse to
grafene krysse hverandre, fordi de begge er \textit{kontinuerlige} (som betyr
at ingen kan ha diskontinuerlige <<hopp>> slik at punktene ikke sammenfaller),
og fordi de starter og stopper på motsatte y-punkter fra $0$ til
$\frac{\pi}{2}$.

Det betyr at mellom $x=0$ og $x=\frac{\pi}{2}$ så \textit{må} det finnes et punkt $x$
slik at $\cos{x} = x$.

\oppgave{5}

\oppgave{6a}
Vi løser dette ved å kansellere $x+2$. Dette kan vi gjøre
fordi $x$ vil aldri være eksakt lik $2$:
\begin{equation*}
\begin{split}
  \lim_{x \to -2}{\frac{x^2+4x+4}{x+2}} = \lim_{x \to -2}{\frac{(x+2)^2}{x+2}}
  = \lim_{x \to -2}{x+2} = \svar{0}
\end{split}
\end{equation*}

\oppgave{6b}
Her har vi ingen singulariteter for $x>0$, så da kan vi utvide brøken med
$\frac{1}{x^5}$:
\begin{equation*}
\begin{split}
  \lim_{x \to \infty}{\frac{x^5+2x^2+12}{3x^5+18x^4+10x^2+18}} =
    \lim_{x \to \infty}{
        \frac{
          1+\frac{2}{x^3}+\frac{12}{x^5}
        }{
          3+\frac{18}{x}+\frac{10}{x^3}+\frac{18}{x^5}
      }}
\end{split}
\end{equation*}
For alle $x>0$ vil $\lim_{x \to \infty}{\frac{1}{x^n}} = 0$, dermed får vi
\begin{equation*}
\begin{split}
  = \lim_{x \to \infty}{\frac{1}{3}} = \svar{\frac{1}{3}}
\end{split}
\end{equation*}
Men hva med $x=0$? Siden $x\to\infty$ så snakker vi om en \textit{énsidet}
grenseverdi, dermed kan vi forenkle ved å si at vi undersøker grensen fra og
med første $x>0$.

\oppgave{6c}
Igjen, vi skriver $x^2-4x+3 = (x-1)(x-3)$ og kansellerer $x-1$, fordi $x$ vil
nærme seg $1$ men aldri bli eksakt lik. Da unngår vi null i nevneren.
\begin{equation*}
\begin{split}
  \lim_{x \to 1}{\frac{(x-1)(x-3)}{x-1}} =
  \lim_{x \to 1}{x-3} = \svar{-2}
\end{split}
\end{equation*}

\end{document}
