\documentclass[a4paper,norsk,12pt]{article}
\usepackage[utf8]{inputenc}

% Oppsett for norsk
\usepackage[norsk]{babel}
\usepackage{times}
\usepackage[T1]{fontenc}
\usepackage{parskip}
\DeclareUnicodeCharacter{00A0}{ }
\newcommand{\strek}{\textthreequartersemdash}

% Andre pakker
\usepackage{oving}
\usepackage{amsmath}
\usepackage{amssymb}
\usepackage{varioref}
\usepackage{subcaption}
\usepackage{units}


\title{MAT100 --- Matematiske metoder 1}
\subtitle{Obligatorisk innlevering 2}
\author{Christian Stigen}
\date{UiS, 9.~oktober, 2015}

\begin{document}
\maketitle

\oppgave{1a}
\paragraph{(1)} Prøver med $y=e^{rt}$,
\begin{align*}
  r^2e^{rt} + 5re^{rt} - 6e^{rt} = 0 \\
  e^{rt}(r^2+5r-6) = e^{rt}(r+6)(r-1) = 0
\end{align*}
%
som gir $D = b^2 - 4ac = 49$. Med $D > 0$ har vi reelle forskjellige røtter
$r_1$ og $r_2$, og siden $e^{rt} \neq 0$ er $r_1=-6$ og $r_2=1$. Den generelle
løsningen er da
%
\begin{align*}
  y & = Ae^{r_1t} + Be^{r_2t} = \svar{Ae^{-6} + Be^t}
\end{align*}

\paragraph{(2)} Prøver igjen med $y=e^{rt}$,
\begin{align*}
  e^{rt}(r^2 + 6r + 9) = e^{rt}(r+3)^2 = 0
\end{align*}
Vi ser at vi har sammenfallede røtter ($D=0$), derfor er én løsning $y_1 =
e^{rt}$. Forsøker vi med $y = ue^{rt}$ får vi den generelle løsningen
\begin{align*}
  y & = (A + Bt)e^{-3t} = \svar{Ae^{-3t} + Be^{-3t}t}
\end{align*}

\oppgave{2}
\textit{Ikke utført}

\oppgave{3}
\paragraph{(a)}
Fart er avstand over tid, og ved høyeste punkt er farten null.
\begin{align*}
  \frac{ds}{dt} = \dot{s} = -32t + 160 & = \unitfrac[0]{m}{s} \\
  t = \nicefrac{160}{32} & = \unit[5]{s} \\
  s = 5(160-16\cdot5) & = \svar{\unit[400]{m}}
\end{align*}

\paragraph{(b)}
Da dette er lavere enn makshøyden, vil raketten passere denne høyden to ganger,
med lik absoluttfart men motsatt fortegn:
\begin{align*}
  s  = 160t-16t^2 & = \unit[256]{m} \\
  -16t^2+160t-256 & = 16(t-8)(t-2) = 0 \\
  t = \unit[2]{s} & \vee t = \unit[8]{s} \\
  \dot{s}(2) = -32\cdot2 + 160\cdot2 & = \svar{\unitfrac[96]{m}{s}}
\end{align*}

\paragraph{(c)}
Akselerasjon er momentanendring av fart:
\begin{align*}
  \ddot{s} = \frac{d\dot{s}}{dt} = \svar{\unitfrac[-32]{m}{s^2}}
\end{align*}

\paragraph{(d)}
Ved $t=\unit[5]{s}$ er raketten i toppunktet. Da er den nøyaktig halvveis,
dermed treffer den bakken ved $\svar{t=\unit[10]{s}}$, og vi ser at
$s(0) = s(10) = \unit[0]{m}$.

\oppgave{4}
\paragraph{(a)}
Ved l'Hôpital,
\begin{align*}
  \lim_{x\to 0}{\frac{3x-\cos{x}}{1}} = 3 - 1 = \svar{2}
\end{align*}

\paragraph{(b)}
\textit{Ikke utført}
\paragraph{(c)}
\textit{Ikke utført}
\paragraph{(d)}
\textit{Ikke utført}

\oppgave{5}
\textit{Ikke utført}

\oppgave{6}
\begin{align*}
  f'(x) = 3x^2 + 1 & = 5 \text{ for } x \in [0,2]\\
  x^2 = \frac{4}{3} & \text{ gir }
  \svar{x = \frac{2}{\sqrt{3}}}
\end{align*}

\end{document}
