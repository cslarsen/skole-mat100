\documentclass[a4paper,norsk,12pt]{article}
\usepackage[utf8]{inputenc}

% Oppsett for norsk
\usepackage[norsk]{babel}
\usepackage{times}
\usepackage[T1]{fontenc}
\usepackage{parskip}
\DeclareUnicodeCharacter{00A0}{ }
\newcommand{\strek}{\textthreequartersemdash}

% Andre pakker
\usepackage{oving}
\usepackage{amsmath}
\usepackage{amssymb}
\usepackage{varioref}
\usepackage{subcaption}
\usepackage{units}


\title{MAT100 --- Matematiske metoder 1}
\subtitle{Obligatorisk innlevering 2}
\author{Christian Stigen}
\date{UiS, 9.~oktober, 2015}

\begin{document}
\maketitle

\oppgave{1a}
\paragraph{(1)} Prøver med $y=e^{rt}$,
\begin{equation*}
  \begin{split}
    r^2e^{rt} + 5re^{rt} - 6e^{rt} = 0 \\
    e^{rt}(r^2+5r-6) = e^{rt}(r+6)(r-1) = 0
  \end{split}
\end{equation*}
%
som gir $D = b^2 - 4ac = 49$. Med $D > 0$ har vi reelle forskjellige røtter
$r_1$ og $r_2$, og siden $e^{rt} \neq 0$ er $r_1=-6$ og $r_2=1$. Den generelle
løsningen er da
%
\begin{equation*}
  \begin{split}
    y & = Ae^{r_1t} + Be^{r_2t} = \svar{Ae^{-6} + Be^t}
  \end{split}
\end{equation*}

\paragraph{(2)} Prøver igjen med $y=e^{rt}$,
\begin{equation*}
  \begin{split}
    e^{rt}(r^2 + 6r + 9) = e^{rt}(r+3)^2 = 0
  \end{split}
\end{equation*}
Vi ser at vi har sammenfallede røtter ($D=0$), derfor er én løsning $y_1 =
e^{rt}$. Forsøker vi med $y = ue^{rt}$ får vi den generelle løsningen
\begin{equation*}
  \begin{split}
    y & = (A + Bt)e^{-3t} = \svar{Ae^{-3t} + Be^{-3t}t}
  \end{split}
\end{equation*}


\end{document}
